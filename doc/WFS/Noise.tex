\section{Sensor Noise}

The wavefront sensor will experience several noise sources that limit the accuracy of the centroid algorithm and the control loop.  Most notably are the discrete sampling of the intensity distribution, the incident photon noise and the detector read out noise.  The latter two are discussed here.  

\subsection{Photon Noise}

Photon noise, also called Shot noise, is due to the stochastic characteristics of light.  Since light behaves as a stream of single particles called photons, the intensity of the source can be considered as an average value with a certain fluctuation.  This is modelled by a Poisson distribution with a standard deviation equal to the square root of the signal's intensity. This noise will vary with the exposure time and the strength of the source.  

\subsection{Readout Noise}

Readout noise is attributed to the CCD or CMOS sensor that is used to convert the photons into an electrical signal.  These devices are not perfect and the conversion and amplification process results in readout noise.  Since this noise is from the electronics, it is present no matter the exposure time, signal intensity etc.  The readout noise is modelled as white Gaussian, with a zero mean and normal distribution with standard deviation $\sigma_r$, and where the noise of each pixel is mutually uncorrelated. 
%Ref: Iterative linear focal-plane wavefront correction article.

\subsection{Measurement Noise}
The combination of photon and readout noise are independent, but both result in errors for the centroid-ing algorithm.  Of interest is the total noise at the output (Equation \ref{TotNoiseVar}).

\begin{equation}
\sigma_0^2 = g^2\sigma_p^2 + g^2\sigma_r^2
\label{TotNoiseVar}
\end{equation}

where $\sigma_0^2$ is the total noise output variance, $\sigma_p^2$ is the photon noise variance and $\sigma_r^2$ is the readout noise variance.  The photon gain of the CCD detector, $g$, relates the number of electrons generated by a photon at the detector.  Since it follows a Poisson distribution, the variance of the photon noise is equal to the mean number of photon events (intensity), $p_i$.  Using the photon gain it can be expressed in terms of the number of output events since $p_i = gp_o$.  This gives the relationship in Equation \ref{NoiseVar}.

\begin{equation}
\sigma_0^2 = gp_o + g^2\sigma_r^2
\label{NoiseVar}
\end{equation}
% Ref: Accuracy analysis of H-S wavefront sensor operated with a fain object

% After noise has been implemented, show figures of PSF with and without noise.