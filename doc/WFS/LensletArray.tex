\section{The Intensity Distribution of a Lenslet Array}
Every intensity distribution of a lenslet is calculated separately by the procedure discussed above in Sections \ref{sec:FrApprox} and \ref{sec:DFT}.
If the incident rays enter on a to big angle, the FFT will be incorrect. To solve this, the average tilt of the incident phase on a lenslet is calculated. From this tilt a shift of the intensity distribution can be calculated. Putting this into a loop we calculate every intensity distribution of every lenslet with the correct coordinates. Then the distributions are interpolated on a grid that represent the Charge-Couple Device (CCD) giving the complete intensity distribution. 

The precise procedure for the calculation of the normalized intensity distribution of a lenslet array goes as followed:
\begin{enumerate}
	\item 
	A spatial grid ($x_t,y_t$)  with the dimension $l_x \times l_y$ of the lenslet array is created to contain the intensity distribution in the focal plane. This grid has the same sample spacings $\Delta x$ and $\Delta y$ as the incident wavefront on the lenslet array.
	\item
	A spatial support gird ($\xi_{sup},\eta_{sup}$) with the dimension $L\times L$ is created, where $L$ can be adjusted to get a high enough sampled diffraction spot per lenslet. The sample spacing is taken as $\Delta x$ because it is a square grid.
	\item
	A spatial grid ($x_{sup},y_{sup}$) with the dimension $\frac{\lambda f}{\Delta x} \times \frac{\lambda f}{\Delta x}$ is created to contain the FFT needed for the calculation of the diffraction spots. The sample spacing can be obtained by $\Delta x_{sup} = \lambda f/L$.  
	\item
	Now for every lenslet the corresponding phase plate $\phi_n(\xi,\eta)$, for $n = 1,2,...,N_{lens}$ where $N_{lens}$ is the number of lenslets, is extracted form the incident phase $\phi(\xi,\eta)$ over the whole lenslet array.
	\item
	To avoid problems with aliasing we want the capture the main dynamics of the diffraction pattern in the middle of the grid ($x_{sup},y_{sup}$). This can be done by removing the average phase tilt $\overline{\nabla \phi_n}(\xi,\eta)$ of the phase $\phi_n(\xi,\eta)$, thus the processed phase plate is
	\begin{equation}
	\hat{\phi}_n(\xi,\eta) = \phi_n(\xi,\eta) - \overline{\nabla \phi_n}(\xi,\eta).
	\end{equation}
	\item
	Still the shift in the focal plane that is caused by the tilt of the phase has to be applied to the spatial grid ($x_{sup},y_{sup}$). To do this the spatial shifts $\sigma_x$ and $\sigma_y$ needs to be computed from the average phase tilt $\overline{\nabla \phi_i}(\xi,\eta)$. This can be done by
	\begin{equation}
	\sigma_x = \frac{\overline{\nabla \phi_n}(\xi,0)}{k}f,~~~\text{and}~~~\sigma_y = \frac{\overline{\nabla \phi_n}(0,\eta)}{k}f.
	\end{equation}
	This relation can be confirmed by the calculation of Equation \eqref{eq:fresnel} with oblique incident wavefronts, which applies as long as the incident angles are small.
	\item
	Now calculate the complex amplitude of each lenslet in the focal plane. This is done by taking the FFT of $P(\xi,\eta)U_i(\xi,\eta)$, where in this case
	\begin{equation}
	U_i(\xi,\eta) = e^{j\hat{\phi}_n(\xi,\eta)}
	\end{equation}
	and $P(\xi,\eta)$ is the pupil function. Then multiplying this by the multiplicative factor from Equation \eqref{eq:fresnel}.
	\item 
	The complex amplitude $U_f(x_{sup},y_{sup})$ is now known in its relative dimensions for every lenslet and the intensity distribution in the right absolute spatial dimensions per lenslet is now calculated by 
	\begin{equation}
	I_f(x,y)=|U_f(x_{sup} - \sigma_x - l_{centre,x},y_{sup} - \sigma_y - l_{centre,y})|^2,
	\end{equation}
	where $l_{centre,x}$ and $l_{centre,y}$ is the center position of the lenslet on the x and y-axis respectively.  
	\item 
	By interpolation the intensity distributions of every lenslet is collected into the spatial grid ($x_t,y_t$) and normalized as
	\begin{equation}
	\tilde{I_t}(x_t,y_t) = \frac{I_t(x_t,y_t)}{||I_t(x_t,y_t)||_\infty}
	\end{equation}
	
\end{enumerate}


