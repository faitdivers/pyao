\documentclass{article}

\usepackage[utf8]{inputenc}
\usepackage[english]{babel}
\usepackage{mathrsfs,amsmath}
\usepackage{float}
\usepackage{subfig}
\usepackage{units}
\usepackage{cite}
\usepackage{slashbox}
\usepackage{listings}
\usepackage{color}

\definecolor{mygreen}{rgb}{0,0.6,0}
\definecolor{mygray}{rgb}{0.5,0.5,0.5}
\definecolor{mymauve}{rgb}{0.58,0,0.82}


\usepackage[a4paper]{geometry}	% enlarges the page margins to standard A4 paper
\usepackage{fancyhdr}		          % header usepackage
\usepackage{graphicx}		         % figures usepackage
\usepackage{mathtools}                       % Used in some equations for displaying text
\usepackage{amsmath}                        % No idea what it is, but seems usefull
\usepackage{amssymb} 		        % symbol for set
\usepackage{mathrsfs}                        %lagrange symbol
\usepackage{float}
\usepackage{epstopdf}

% Puts captions of tables on top
\floatstyle{plaintop}
\restylefloat{table}

% Puts captions in bold
\captionsetup{labelfont=bf}

\begin{document}
\section{Introduction}
This manual is intended as reference guide to show you around the wavefront generation code of Adaptive Optics in Python. This manual will comprise of a description of the classes as well as their methods and functions. As far as is reasonably necessary, an example is included.

As current, the WFG part includes two classes. One class on the Zernike aberrations. The second class is on the stochastic behaviour of the the wavefront, as for example Kolmogorov behaviour.

\newpage
\lstset{language=Python} 
% Zernike static methods
\section{Zernike Functions}
\subsection{zernike}
Z = zernike(rho, theta,u,v = None):

Returns


\subsection{zernikeIndex}
u,v = def zernikeIndex(i):


% Zernike dynamic class
\section{ZernikeWave Class}
\subsection{Constructor}

\subsection{addMode}

\subsection{changeModeWeight}

\subsection{removeMode}

\subsection{modeExists}

\subsection{getModes}

\subsection{getWeights}

\subsection{createWavefront}

\subsection{decomposeWavefront}

\subsection{plotMode}

\subsection{plotWavefront}

% Phasescreen dynamic class
\section{PhaseScreen Class}
\subsection{Constructor}

\subsection{setType}

\subsection{getType}

\subsection{setParams}

\subsection{getParameters}

\subsection{createWavefront}

\subsection{kolmogorov}

\subsection{vonkarman}

\subsection{plotWavefront}

% Static support function on wfg
\section{Support functions}
\subsection{kroneckerDelta}

\subsection{gamma2}

\subsection{cart2pol}

\subsection{pol2cart}

\subsection{createGrid}

\subsection{circ}


\end{document}